% !TeX program = lualatex
% !TeX encoding = UTF-8
% !TeX spellcheck = fr_FR

\documentclass[a4paper, french, twoside]{article}
\usepackage{polyglossia}
\setdefaultlanguage{french}
\usepackage{fontspec}
\defaultfontfeatures{Ligatures={Common, TeX}}
\usepackage[pdfusetitle, colorlinks=true]{hyperref}
\usepackage{mathtools, amssymb}
\usepackage{xcolor}
\usepackage{graphicx}
\usepackage[left=2cm, right=2cm, top=2cm, bottom=2cm]{geometry}
\usepackage{enumitem}
\setlist{noitemsep}

\definecolor{airforceblue}{rgb}{0.36, 0.54, 0.66}
\definecolor{charcoal}{rgb}{0.21, 0.27, 0.31}
\definecolor{ube}{rgb}{0.53, 0.47, 0.76}
\definecolor{violet}{rgb}{0.2, 0.2, 0.6}
\definecolor{blue(ncs)}{rgb}{0.0, 0.53, 0.74}
\definecolor{amber}{rgb}{1.0, 0.49, 0.0}
\definecolor{ao}{rgb}{0.0, 0.5, 0.0}

\usepackage[procnames]{listings}
\lstset{language=Python ,
mathescape=true,
frame=trBL,
numbers=none,
lineskip=0.01mm,
backgroundcolor=\color{white},
basicstyle=\scriptsize,
keywordstyle=\color{blue(ncs)},
commentstyle=\color{amber},
stringstyle=\color{ao},
identifierstyle=\color{black},
breaklines=true,
inputpath={C:/Users/Adrien Dubois/Desktop/TIPE/2-Code/}}
	
\title{TIPE : Code}
\author{Adrien Dubois - N°candidat : 51771}
\date{\today}


\begin{document}
\maketitle
\tableofcontents

\section{Lecture PDB et génération des exemples}

\subsection{Extraction du fichier PDB}
\lstinputlisting[lastline=79]{extract_PDB.py}
\subsection{Génerer les branches}
\lstinputlisting{generer_nuage.py}
\subsection{Modifier les protéines}
\lstinputlisting{generer_protein.py}
\subsection{Arbre couvrant}
\lstinputlisting{arbre_couvrant.py}
\subsection{Générer les graphes}
\lstinputlisting{creer_graph.py}
\subsection{Exemples protéines}
\lstinputlisting[firstline=83]{extract_PDB.py}

\section{Définition des classes}

\subsection{Branches}
\lstinputlisting[firstline=1,lastline=124]{cas_simple_nuage.py}
\subsection{Graphes}
\lstinputlisting[firstline=1,lastline=37]{graphisomorphism.py}
\subsection{Protéines}
\lstinputlisting[firstline=1,lastline=87]{def_protein.py}

\section{Isomorphisme et sous-isomorphisme}

\subsection{Isomorphsime sur les graphes}
\lstinputlisting[firstline=39]{graphisomorphism.py}
\subsection{Isomorphsime sur les protéines}
\lstinputlisting[firstline=88]{def_protein.py}
\subsection{Sous-isomorphisme sur les protéines}
\lstinputlisting[firstline=14,lastline=68]{subisom_prot.py}
\subsection{Informations et temps d'exécution}
\lstinputlisting{isom_prot.py}
\lstinputlisting[firstline=68,lastline=73]{subisom_prot.py}

\section{Fonctions auxiliaires}
\lstinputlisting{geometrie_et_aux.py}

\section{Calcul des coefficients}
\lstinputlisting[firstline=126,lastline=154]{cas_simple_nuage.py}
\lstinputlisting[firstline=91]{subisom_prot.py}

\section{Affichage}
\subsection{Graphique complexité}
\lstinputlisting{plot_graph_complexite_isom_brutes.py}
\subsection{Affichage branches}
\lstinputlisting{plot_distlines.py}
\subsection{Affichage protéines}
\lstinputlisting{plot_protein.py}
\subsection{Affichage comparaison protéines}
\lstinputlisting{plot_sub.py}

\end{document}