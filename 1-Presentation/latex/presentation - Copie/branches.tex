\section{Comparaison de branches sans ramifications}
\begin{frame}{Format PDB et représentation}
    \footnotesize
    \begin{figure}[!htb]
        \centering
        \includegraphics[width=9cm]{lignes_pdb}
        \caption{\label{PDB_tuto:http://acces.ens-lyon.fr/}Lignes d'un fichier PDB \footnote{source : http://acces.ens-lyon.fr/}}
    \end{figure}
    \begin{minipage}{0.45\textwidth}%
    \begin{center}
        \underline{Données}
    \end{center}
    \begin{tabular}{ccc}
    position des atomes & $\rightarrow$ & OK \\
    type des atomes & $\rightarrow$ & OK \\
    liaisons & $\rightarrow$ & moyennes
    \end{tabular}
    \end{minipage}%
    \hfill
    \begin{minipage}{0.45\textwidth}%
    \begin{figure}[!htb]
        \centering
        \includegraphics[width=4cm]{protein_peuliaisons_pres}
    \end{figure}
    \end{minipage}%
\end{frame}

\subsection{Notion de branche}
\begin{frame}{Branche sans ramification}
    \begin{minipage}{0.5\textwidth}%
        \begin{figure}[!htb]
            \centering
            \includegraphics[width=5.5cm]{protein_branche}
        \end{figure}
    \end{minipage}%
    \hfill
    \begin{minipage}{0.4\textwidth}%
        On sélectionne une branche indépendamment du reste de la 
        protéine et du type des sommets
    \end{minipage}%
    \newline \newline \newline
    \underline{en python} : 
    \begin{itemize}
        \item type : \textcolor{airforceblue}{Coord} : $float\ *\ float\ *\ float$
        \item type : \textcolor{airforceblue}{Branche} : Coord list
            \newline \underline{ex} : $B = [(0,0,0), (1,1,1), (1.5,1,2)]$
    \end{itemize}
\end{frame}

\subsection{Coefficient}
\begin{frame}{Proximité de deux branches}
    \footnotesize
    \begin{minipage}{0.47\textwidth}%
    On considère 2 branches \textcolor{blue}{B1 en bleu} et \textcolor{violet}{B2 en violet} :
    \begin{itemize}
    \item même nombre d'atomes
    \item atomes ordonnés
    \item \underline{coloration distances}:
    du vert au rouge, moins élevé au plus élevé
    \end{itemize}
    \end{minipage}%
    \hfill
    \begin{minipage}{0.51\textwidth}%
        \begin{figure}[!htb]
            \includegraphics[width=5.3cm]{dlines_colored}
        \end{figure}
    \end{minipage}%
\end{frame}

\begin{frame}{Coefficient de proximité de branches}
    \footnotesize
    Soit $B_1 = [M_1,\cdots,M_n]$ et $B_2 = [N_1,\cdots,N_n]$ des branches
    \newline On pose, 
    \begin{align*}
    \forall i \in [\![1,n-1]\!],\ & l_i = M_iM_{i+1} \text{ et } l_i' = N_iN_{i+1}\\
    & d_i = max(l_i,l_i') \\
    & \theta_i = angle(  \overrightarrow{M_iM_{i+1}} ,   \overrightarrow{N_iN_{i+1}} ) 
    \end{align*}
    \begin{equation*}
        C_{dist} = \frac{\displaystyle\sum_{i=0}^{n-1} | \sin(\theta_{i}) | d_i}{\displaystyle\sum_{i=0}^{n-1} d_i} \qquad  
        C_{angle} = \frac{\displaystyle\sum_{i=0}^{n-1} |l_i' - l_i|}{\displaystyle\sum_{i=0}^{n-1} d_i}
    \end{equation*}
    \begin{equation*}
        R_{dist} = \frac{1}{1 + C_{dist}} \qquad R_{angle} = \frac{1}{1 + C_{angle}}
    \end{equation*}
    \newline Finalement, on définit : \newline
    \begin{equation*}
        \boxed{R = \frac{R_{dist}+R_{angle}}{2}}
    \end{equation*}
\end{frame}

\begin{frame}{Valeur du coefficient sur exemples}
    \begin{multicols}{2}
        \begin{figure}
            \centering
            \includegraphics[width=3.5cm]{rcoeff_dist059_ang068_tot064}
        \end{figure}
    \vspace*{0.2cm}
        \begin{align*}
            R_{dist} = &0.59 \quad R_{angle} = 0.68 \\
            &\boxed{R = 0.64}
        \end{align*}
    \end{multicols}
    \begin{multicols}{2}
        \begin{figure}
            \centering
            \includegraphics[width=3.5cm]{rcoeff_dist094_ang097_tot095}
        \end{figure}
    \vspace*{0.2cm}
            \begin{align*}
                R_{dist} = &0.94 \quad R_{angle} = 0.97 \\
                &\boxed{R = 0.95}
            \end{align*}
    \end{multicols}
\end{frame}