% !TeX program = lualatex
% !TeX encoding = UTF-8
% !TeX spellcheck = fr_FR

\documentclass[a4paper, french, twoside]{article}
\usepackage{polyglossia}
\setdefaultlanguage{french}
\usepackage{fontspec}
\defaultfontfeatures{Ligatures={Common, TeX}}
\usepackage[pdfusetitle, colorlinks=true]{hyperref}
\usepackage{mathtools, amssymb}
\usepackage{xcolor}
\usepackage{graphicx}
\usepackage[left=2cm, right=2cm, top=2cm, bottom=2cm]{geometry}
\usepackage{enumitem}
\setlist{noitemsep}
	
\title{Docs exemple}
\author{Adrien Dubois - N° Candidat : 51771}
\date{\today}


\begin{document}
\maketitle
\tableofcontents

\part{Une partie}
Ici commence une partie\dots{} avec une liste à puces numérotée où un item fait référence à un autre :
	\begin{enumerate}
	\item\label{it:point} premier élément
	\end{enumerate}

\section{Une section}
Traçons la courbe $y=x^2$ pour $x$ variant entre 0 et 2.\footnote{Ceci est une note de bas de page.}

\subsection{Une sous-section}

\subsection{Une autre sous-section}\label{sec:unereference}
Un groupe d'équations avec \texttt{align} et référence. Le résultat est l'équation~\eqref{eq:prés}
	\begin{align}
	\langle\mathcal{P}\rangle
	&=\iint_S\langle\vec{\Pi}\rangle.d\vec{S}\\
	&=\frac{\omega^4p_0^2}{12\pi\varepsilon_0c^3}\label{eq:prés}
	\end{align}

Un résultat avec encadré :
	\begin{align}
	\vec{r}=x\,\vec{u}_x+y\,\vec{u}_y
	\quad\Rightarrow\quad
	\boxed{
	R=\lVert\vec{r}\rVert=\sqrt{x^2+y^2}
	}
	\quad\text{car $R\in\mathbb{R}^+=[0,+\infty[$}
	\end{align}


\section{Une autre section}
Voir paragraphe \ref{sec:unereference}. La relation~\eqref{eq:matrice} contient une matrice :
    \begin{align}
    A =
    \begin{pmatrix}
    1 & a \\
    4 & -6
    \end{pmatrix}
    \quad\text{avec $a\geqslant 0$}\label{eq:matrice}
    \end{align}
et un tableau :
    \begin{center}
    \begin{tabular}{l|c|c}
    Colonne 1  & Colonne 2 & Colonne 3 \\
    \hline
    Intitulé 1 & \textcolor{red}{1} & \textcolor{blue}{2} \\
    Intitulé 2 & \textcolor{green}{3} & 4
    \end{tabular}
    \end{center}

\end{document}
